\def\OPTIONConf{0}%
\def\OPTIONArxiv{0}%
%
\documentclass{llncs}
\usepackage{joshuadunfield}
\usepackage{boxedminipage}
\usepackage{goodcharter} %% XXX -- Check that ETAPS permits this font!
%\usepackage{euler}
\usepackage{mathtools}
\usepackage{semantic}   % Tools for typesetting PL semantics
\usepackage{braket}     % Easy angle-bracket notation
\usepackage{mathpartir} % Used to typeset blocks of inference rules
\usepackage{rotating} % for sidewaysfigure
%\usepackage{proof}
\usepackage{pdflscape}
\usepackage{fancyvrb} % to use \verb in footnotes
\usepackage{stmaryrd}
\usepackage{amssymb}
%\usepackage{amsmath,amsthm,amssymb}
%\usepackage{thmtools,thm-restate}


% Marvosym
\let\MathRightArrow\Rightarrow % save original definition of \Rightarrow
\usepackage{marvosym} % feloniously overrides \Rightarrow
\def\Rightarrow{\MathRightArrow}

%% \declaretheorem[style=mytheoremstyle]{theorem}
%% \declaretheorem[style=mytheoremstyle]{property}
%% %\declaretheorem[style=mytheoremstyle, sibling=theorem]{lemma}
%% \declaretheorem[style=mytheoremstyle]{lemma}
%% \declaretheorem[style=mytheoremstyle, sibling=lemma]{corollary}
%% \declaretheorem[style=mytheoremstyle, sibling=lemma]{conjecture}
%% \declaretheorem{example}
%% \makeatletter
%% \declaretheorem[style=mytheoremstyle, sibling=lemma, 
%% %             postheadhook={%
%% %               envname is `\thmt@envname'; %
%% %               thmname is `\thmt@thmname'; %
%% %               optarg is `\thmt@optarg'; %
%% %%               innercounters are `\thmt@innercounters'.
%% %             }
%%        ]{proposition}
%% \makeatother
%% \declaretheorem{remark}
%% \declaretheorem[style=mytheoremstyle]{definition}


%% \theoremstyle{remark}

\newcommand{\calc}{$\lambda_{\textsf{nom}}$\xspace}

\newcommand{\Label}[1]{\label{#1}} % XXX
\newcommand{\FLabel}[1]{\label{#1}} % XXX
\newcommand{\D}{\mathcal{D}}
\newcommand{\Ss}{\mathcal{S}}
\newcommand{\derives}{::}
\newcommand{\satisfactory}{~\textsf{satisfactory}}
%\newcommand{\wf}{~\textsf{wf}}

%% \newtheorem{thm}{Theorem}[section]
%% \newtheorem*{thm*}{Theorem}
%% \newtheorem{lem}[thm]{Lemma}
%% \newtheorem{conj}[thm]{Conjecture}
%% \newtheorem{prop}[thm]{Proposition}
%% \newtheorem*{cor}{Corollary}
%% \newcommand{\thmref}[1]{Theorem~\ref{thm:#1}}
%% \newcommand{\thmreftwo}[2]{Theorems \ref{thm:#1} and~\ref{thm:#2}}
%% \newtheorem{defn}[thm]{Definition}

\newcommand{\type}{\star}
\newcommand{\sort}{\gamma}
\newcommand{\namesort}{\boldsf{Nm}}
\newcommand{\namesetsort}{\boldsf{NmSet}}
\newcommand{\unitsort}{\boldsf{1}}

\newcommand{\Nmsp}{\tyname{Nmsp}}

\newcommand{\xwfeff}{\text{wf-effects}}
\newcommand{\wfeff}{~\xwfeff}
\newcommand{\xwfname}{\text{wf-name}}
\newcommand{\wfname}{~\xwfname}

\newcommand{\xwf}{\text{wf}}
\newcommand{\wf}{~\xwf}

\newcommand{\xwfgraph}{\xwf}
\newcommand{\wfgraph}{~\xwfgraph}

% \newcommand{\ixf}{\boldsf{f}}

\newcommand{\namevar}{\nu}

\newcommand{\Unit}{\tyname{unit}}
\newcommand{\unitty}{\Unit}
\newcommand{\Nat}{\tyname{nat}}
\newcommand{\natty}{\Nat}
\newcommand{\unit}{\textvtt{()}}
\newcommand{\unitexp}{\unit}

\newcommand{\unitindex}{\unit}

\newcommand{\runonboldsf}{\sffamily\bfseries\selectfont}
\newcommand{\boldsf}[1]{\text{\runonboldsf #1}}

\newcommand{\xF}{\boldsf{F}}
\newcommand{\F}{\xF\,}
\newcommand{\xU}{\boldsf{U}}
\newcommand{\U}{\xU\,}

% Read and Write channel types
\newcommand{\xRd}{\boldsf{Rd}}
\newcommand{\Rd}{\xRd\,}
\newcommand{\xWr}{\boldsf{Wr}}
\newcommand{\Wr}{\xWr\,}

% Modes
\newcommand{\Rm}{\textsf{R}}
\newcommand{\Wm}{\textsf{W}}
\newcommand{\Vm}{\textsf{V}}

\newcommand{\disj}{\mathop{\bot}}

\newcommand{\trueprop}{\boldsf{tt}}
\newcommand{\andpropsym}{\boldsf{and}}
\newcommand{\andprop}{\mathrel{\andpropsym}}
\newcommand{\impty}{\supset}

% \Univ currently deprecated
\newcommand{\Univsym}{\Pi}
% \Univ currently deprecated
\newcommand{\Univ}[1]{\Univsym{#1}.\,}

% Currently, using \All for both type variables (\All{\alpha : K}
% and index variables (\All{ea : \sort})
\newcommand{\Allsym}{\forall}
\newcommand{\All}[1]{\Allsym{#1}.\,}

% Name function arrow (name in scope at both the type and term level!)
% \Namearr{a}{... @> ...}
\newcommand{\Namearrsym}{\aleph}
\newcommand{\Namearr}[2]{\Namearrsym{#1}:{#2} \arr}

\newcommand{\namearrintro}[1]{\Namearrsym{#1}.\;}
\newcommand{\namearrelimsym}{\texttt{@}}
\newcommand{\namearrelim}[2]{{#1} \mathop{\namearrelimsym} {#2}}

\newcommand{\tyapp}[2]{{#1}\tbrack{#2}}
% \newcommand{\idxapp}[2]{{#1}\tbrack{#2}}
\newcommand{\idxapp}[2]{{#1}({#2})}

\newcommand{\Split}[4]{\keyword{split}\Lparen{#1}, {#2}.{#3}.{#4}\Rparen}
\newcommand{\Case}[5]{\keyword{case}\Lparen{#1}, {#2}.{#3}, {#4}.{#5}\Rparen}

\newcommand{\inj}[1]{\keyword{inj}_{#1}\,}
\newcommand{\Inj}[1]{\inj{#1}}

\newcommand{\Scope}[2]{\keyword{scope}\Lparen{#1}, {#2}\Rparen}

\newcommand{\Sig}{\mathcal{S}}


%% Math ligatures (thanks to the semantic package) that make it
%% easier to typeset math using readable LaTeX text.
%\mathlig{|-->}{\longmapsto}
\mathlig{::=}{\bnfas}
\mathlig{|}{\;|\;}
% \mathlig{[[}{\mbsf{[}}
% \mathlig{]]}{\mbsf{]}}
\mathlig{[[}{\texttt{\upshape[}}
\mathlig{]]}{\texttt{\upshape]}}
\mathlig{**}{\times}
\mathlig{|>}{\rhd}
\mathlig{->}{\arr}
\mathlig{-->}{\impty}
\mathlig{--->}{\longrightarrow}
\mathlig{=>}{\Rightarrow}
\mathlig{*!}{\boldsf{!}}
\mathlig{||}{\mathrel{|\!|}}
\mathlig{;;}{\mathrel{;}}
\mathlig{*&}{\mathrel{\&}}
\mathlig{*||}{\mathrel{\&}}

\mathlig{@>}{\arr_{\namesort}}


\newcommand{\eff}[2]{{#1} |> {#2}}
\newcommand{\effseqsym}{\mathsf{then}}
\newcommand{\effseq}{\,\effseqsym\,}

\newcommand{\effcoalsym}{\mathsf{after}}
\newcommand{\effcoal}{\,\effcoalsym\,}

\newcommand{\Lparen}{\texttt{(}}
\newcommand{\Rparen}{\texttt{)}}
% \newcommand{\Pair}[2]{\Lparen{#1}\texttt{,}{#2}\Rparen}

%\mathlig{unit}{\unitty}


% Bidirectional typing

\newcommand{\chkcolor}{dBlue}
\newcommand{\syncolor}{dRed}
\newcommand{\isyncolor}{dGreen}
\newcommand{\chk}{\mathrel{\mathcolor{\chkcolor}{\Leftarrow}}}
\newcommand{\uncoloredsyn}{{\Rightarrow}}
\newcommand{\syn}{\mathrel{\mathcolor{\syncolor}{\uncoloredsyn}}}
\newcommand{\RRightarrow}{\mathrel{\mathrlap{\Rightarrow}\mkern2mu\Rightarrow}}
\newcommand{\isyn}{\mathrel{\mathcolor{\isyncolor}{\RRightarrow}}}
\mathlig{<<}{\langle}
\mathlig{>>}{\rangle}
\mathlig{==>}{\syn}
\mathlig{<==}{\chk}
%\mathlig{||}{\Downarrow}
\mathlig{!!}{\Downarrow}
% \mathlig{..}{\epsilon}

\newcommand{\e}{\epsilon}
\newcommand{\hist}{\mathcal{H}}

\newcommand{\ambns}{M}

\newcommand{\disjoint}{\mathrel{\bot}}

\newcommand{\tbrack}[1]{\texttt{\upshape[}{#1}\texttt{\upshape]}}

\newcommand{\rootname}{\textvtt{root}}

%\newcommand{\tv}{tv}
\newcommand{\Mv}{V}
\newcommand{\ntevalsym}{\Downarrow_{\mathsf{M}}}
\newcommand{\nteval}{\mathrel{\ntevalsym}}

\newcommand{\xrefv}{\keyword{ref}}
\newcommand{\xthunk}{\keyword{thunk}}
\newcommand{\xunthunk}{\keyword{unthunk}}
\newcommand{\xname}{\keyword{name}}
%\newcommand{\xName}{\tyname{Name}}
\newcommand{\xName}{\tyname{Nm}}

\newcommand{\xRef}{\tyname{Ref}}
%\newcommand{\xThunk}{\tyname{Thunk}}
\newcommand{\xThunk}{\tyname{Thk}}
\newcommand{\xUnthunk}{\tyname{Unthunk}}

\newcommand{\refv}[1]{\xrefv\;{#1}}
\newcommand{\thunk}[1]{\xthunk\;{#1}}
\newcommand{\unthunk}[1]{\xunthunk\;{#1}}
\newcommand{\name}[1]{\xname\;{#1}}
\newcommand{\Name}[1]{\xName\tbrack{#1}}


\newcommand{\Ref}[1]{\xRef\tbrack{#1}\,}
\newcommand{\Thk}[1]{\xThunk\tbrack{#1}\,}
\newcommand{\Unthk}[1]{\xUnthunk\tbrack{#1}\,}
\newcommand{\Thunk}[2]{\keyword{thunk}\Lparen{#1},{#2}\Rparen}
\newcommand{\Unthunk}[1]{\keyword{unthunk}\Lparen{#1}\Rparen}
\newcommand{\Refe}[2]{\keyword{ref}\Lparen{#1},{#2}\Rparen} % Ref expression form

\newcommand{\List}[3]{\tyname{List}\tbrack{#1;#2}~{#3}}
\newcommand{\Art}[2]{\tyname{Art}\tbrack{#1}~{#2}}
\newcommand{\Int}[0]{\tyname{Int}}

\let\Force\undefined
\newcommand{\Force}[1]{\keyword{force}\Lparen{#1}\Rparen}
\newcommand{\Get}[1]{\keyword{get}\Lparen{#1}\Rparen}

\newcommand{\Ns}[2]{\keyword{ns}\Lparen{#1}\Rparen\{#2\}}

\newcommand{\Ret}[1]{\keyword{ret}\Lparen{#1}\Rparen}
\newcommand{\Let}[3]{\keyword{let}\Lparen{#1},{#2}.{#3}\Rparen}
\newcommand{\LetBang}[3]{\keyword{let!}\Lparen{#1},{#2}.{#3}\Rparen}

% Channel read and Channel write expressions
\newcommand{\rdexp}[1]{\textsf{rd}\,#1}
\newcommand{\wrexp}[2]{\textsf{wr}\,#1<-#2}
\newcommand{\nuexp}[2]{\nu#1.\,#2}

\newcommand{\xstoretype}{\textsf{has-store-type}}
\newcommand{\storetype}{\;\xstoretype\;}
\newcommand{\xisname}{\textsf{is-name}}
\newcommand{\isname}{\;\xisname}

\newcommand{\PreSt}[3]{{#1}\vdash^{#2}_{#3}}

\newcommand{\xterminal}{\textsf{terminal}}
\newcommand{\terminal}{\;\xterminal}

\newcommand{\nscons}[2]{\left[\!\left[ #1~(#2) \right]\!\right]} % Namespace/Name cons operation
% \newcommand{\nsEval}[3]{\vdash t\;p !! q}
% use \nteval (infix) instead

\newcommand{\emptygraph}{\varepsilon}
\newcommand{\emptyctxt}{\varepsilon}

\newcommand{\bnfheader}[1]{\multicolumn{4}{@{}l}{\text{#1}}}
\newenvironment{sbnfarray}{\begin{tabular}{c}\begin{array}[t]{lr@{~~}c@{~~~}ll}}{\end{array}\end{tabular}}
\newenvironment{bnfarray}{\begin{tabular}{c}\begin{array}[t]{r@{~~}c@{~~~}ll}}{\end{array}\end{tabular}}
\newenvironment{xbnfarray}{\begin{tabular}{c}\begin{array}[t]{r@{~~}c@{~}l@{~~}l}}{\end{array}\end{tabular}}

\newcommand{\DHasEffects}[3]{#1~\textsf{reads}~#2~\textsf{writes}~#3}
\newcommand{\DByHasEffects}[4]{#1~\textsf{by}~#2~\textsf{reads}~#3~\textsf{writes}~#4}
\newcommand{\mergeRds}{\cup}
\newcommand{\joinWrs}{\mathrel{\ast}}
\newcommand{\joinGrs}{\mathrel{\ast}}
\newcommand{\mergeGrs}{\cup}
\newcommand{\restrict}[2]{\textsf{restrict}\lfloor{#1}\rfloor_{#2}}
\newcommand{\dom}[1]{\textsf{dom}(#1)}
\newcommand{\update}[4]{#1\{#2 \mapsto #3\} = #4}
%\newcommand{\disjoint}{\cupplus}

%\newcommand{\Dee}{\mathcal{D}}
%\newcommand{\D}{\mathcal{\Dee}}

\newcommand{\te}{e_{\mathsf{terminal}}}

\newcommand{\nfap}[3]{{\color{blue}{}^{#1}_{#2}}{\color{blue}\big[} #3 {\color{blue}\big]}}

%%% Local Variables: 
%%% mode: latex
%%% TeX-master: types
%%% End: 

\newcommand{\msf}[1]{\ensuremath{{\mathsf {#1}}}}
\newcommand{\mtt}[1]{\ensuremath{\mathtt {#1}}}
\newcommand{\mcal}[1]{\ensuremath{\mathcal {#1}}}
\newcommand{\val}{\msf{val}}
\newcommand{\sn}{\msf{sn}}

\newcommand{\tn}{\textnormal}
\newcommand{\codeb}[1]{\textsf{#1}}
\newcommand{\hash}{\ensuremath{{\cal H}}}
\newcommand{\adv}{\ensuremath{{\mathcal A}}\xspace}
\newcommand{\Adv}{\adv}
\newcommand{\A}{\adv}
\newcommand{\samples}{\overset{\$}{\leftarrow}}
\newcommand{\SA}{\msf{SA}}
% SaUCy specific
\newcommand{\SaUCy}{\xspace{\msf{SaUCy}}\xspace}
\newcommand{\leak}{\color{blue}\mtt{leak}}
\newcommand{\eventually}[1]{\color{blue}\mtt{leak}}
\newcommand{\poly}{\textnormal{poly}}
\newcommand{\negl}{\textnormal{negl}}
\newcommand{\execUC}[4]{\mtt{execUC}({#1},{#2},{#3},{#4})}
\newcommand{\chan}[1]{{\underline{\smash{\msf{#1}}}}}
\newcommand{\functionality}[1]{\ensuremath{{\cal F}_{\textnormal{\msf{{#1}}}}}}
\renewcommand{\F}{\functionality}
\newcommand{\G}[1]{{\cal G}_{\textnormal{\tiny {\uppercase{#1}}}}}
\newcommand{\GG}[1]{{\overline {\cal G}}_{\textnormal{\tiny {\uppercase{#1}}}}}
\renewcommand{\C}[1]{{\cal C}_{\textnormal{\tiny {\uppercase{#1}}}}}
\renewcommand{\P}{\ensuremath{\mathcal P}}
\newcommand{\Ps}{\ensuremath{\{\mathcal{P}_i\}_{i \in [N]}}}
\renewcommand{\S}{\ensuremath{\mathcal S}}
\newcommand{\env}{\Z}
\newcommand{\Z}{{\mcal{E}}}
\newcommand{\R}{{\mathcal R}}

\usepackage{enumitem}

\newcommand{\mc}[1]{\mathcal{{#1}}}
\newcommand{\parheader}[1]{\noindent\emph{\textbf{{#1}.}}}

\title{SaUCy: Super Amazing Universal ComposabilitY}
\author{Kevin Liao}
\institute{}

\begin{document}

\maketitle

\begin{abstract}
\ldots
\end{abstract}

\section{Introduction}

Proving that a protocol is secure is an essential component in cryptographic
protocol design. To do so, we must rigorously define what security means, and
then demonstrate that the protocol lives up to the definition. First-cut security
definitions for cryptographic tasks consider only a standalone execution of the
protocol. While this simplifies analyses, these definitions are insufficient
when analyzing the protocol in more complex contexts. Extended security
definitions have been formulated to remedy these drawbacks, but are often
complex and ad hoc.

The Universal Composability (UC) framework by
Canetti~\cite{canetti2001universally} takes an alternative approach---it is a
general framework for reasoning about the security of practically any
cryptographic protocol in a unified and systematic way. The idea is to formulate
a security definition and analyze the protocol as in the standalone model, but
to additionally show that the protocol composes securely with arbitrary other
protocols, i.e.,\ the protocol is secure in \emph{any context}. The benefit of
composability is that complex, UC-secure protocols can then be modularly
constructed from smaller UC-secure ``building blocks''.

While universally composable security is a powerful guarantee, elaborating UC
proofs is highly nontrivial. They are notoriously complex and exist primarily as
``pen-and-paper'' proofs, which makes many of them essentially
unverifiable. Thus, the goal of this work is to place the UC framework on a
proper analytic foundation by developing computer-aided tools and techniques for
reasoning about UC security.\smallskip

\parheader{Contributions} The main contributions of this work are the
following:
\begin{enumerate}
\item We design and implement a programming language called the Interactive
Lambda Calculus (ILC for short) for constructing algorithmic entities within the
UC execution model.
\item We implement the UC execution model in ILC along with a ``library'' of
example algorithmic entities (ideal functionalities).
\item We build a compiler for translating ILC programs into
EasyCrypt~\cite{barthe2011computer} modules, which allows us to construct
cryptographic proofs using EasyCrypt's interactive theorem proving facilities.
\end{enumerate}

\parheader{Organization} We first provide background on the UC framework in Section~\ref{sec:background}.

\section{Background}\label{sec:background}

In the Universal Composability framework, the security of a protocol is
analyzed \emph{in vitro} as a standalone protocol for simplicity, but
security \emph{in vivo}, that is, in realistic settings where the protocol can
run concurrently with arbitrarily many (even adversarially controlled)
protocols, is guaranteed by showing that security is preserved under a general
composition operation on protocols.

Demonstrating that a protocol meets its security definition in the standalone
model goes back to the real/ideal paradigm introduced
in~\cite{goldreich1987play}. First, envision an \emph{ideal protocol} $\phi$
(running with parties $\phi_1, \ldots, \phi_n$) that is secure by definition for
carrying out a cryptographic task. In the ideal protocol, each party hands their
input to an incorruptible trusted party, called an \emph{ideal functionality}
$\mc{F}$. The ideal functionality then performs the task locally and hands to
each party their prescribed output. We can think of the ideal protocol as a
formal specification for the security requirements of the task.

Of course, in the real world, such a trusted party does not exist! Since a real
world protocol $\pi$ (running with parties $\pi_1, \ldots, \pi_n$) cannot rely
on the ideal functionality to carry out the task securely, we would like to show
that $\pi$ \emph{emulates} the ideal protocol $\phi$, which does rely on the
ideal functionality. Informally, we show this by proving that whatever
``damage'' any adversary $\mc{A}$ can do to break $\pi$, there exists an
adversary $\mc{S}$ (known as a simulated adversary, or simulator) that can do
the same ``damage'' to break $\phi$. However, since $\phi$ is secure by
definition, the ``damage'' done by $\mc{A}$ does not break the security of
$\pi$. We will begin to formalize this idea shortly.

The above setup is sufficient for proving security in the standalone model,
which allows for secure \emph{sequential} composition. However, to generalize
composition to the \emph{concurrent} setting, the UC framework includes an
additional entity known as the environment $\mc{Z}$. Intuitively, the
environment represents everything external to the protocol in consideration---it
can hand inputs to protocol parties and interact freely with the adversary throughout the
protocol execution. Essentially, the environment acts as an outside observer,
and its job is to determine whether it is interacting with the real world setup
or the ideal world setup by examining what ``damage'' an adversary can do to
break the protocol. Similar to the security definition described in the
standalone model, if no $\mc{Z}$ can distinguish between the real world setup
(with $\pi$ and $\mc{A}$) and the ideal world setup (with $\phi$ and $\mc{S}$),
then we say that $\pi$ \emph{UC-emulates} $\phi$ and is
thus \emph{UC-secure}. Figure~\ref{fig:uc-experiment} shows the real world and
ideal world setups in the UC experiment.

\begin{figure}[!htb]
\centering
\begin{minipage}{.5\textwidth}
  \centering
  \includegraphics[width=\textwidth]{real-setup}
\end{minipage}%
\begin{minipage}{.5\textwidth}
  \centering
  \includegraphics[width=\textwidth]{ideal-setup}
\end{minipage}
\caption{Real world setup (left) and ideal world setup (right) in the UC experiment.}
\label{fig:uc-experiment}
\end{figure}

Having described the two setups in the UC experiment, we define the random
variable $\textsc{EXEC}_{\pi, \mc{A}, \mc{Z}}$ to be the output of $\mc{Z}$ in
the real world execution and the random variable
$\textsc{EXEC}_{\phi, \mc{S}, \mc{Z}}$ to be the output of $\mc{Z}$ in the ideal
world execution. Additionally, let $k$ be a security parameter.

\begin{definition}[UC Protocol Emulation]
Let $\pi$ and $\phi$ be protocols. We say that $\pi$ UC-emulates $\phi$ if for
any adversary $\mc{A}$ there exists an
adversary $\mc{S}$ such that for any environment $\mc{Z}$, we have:
\begin{equation*}
\textsc{EXEC}_{\phi, \mc{S}, \mc{E}} \approx \textsc{EXEC}_{\pi, \mc{A}, \mc{E}},
\end{equation*}
\noindent i.e.,\ the statistical difference between
$\textsc{EXEC}_{\pi, \mc{A}, \mc{Z}}$ and $\textsc{EXEC}_{\phi, \mc{S}, \mc{Z}}$
is negligible in $k$.
\end{definition}

A more formal definition can be found in the full UC
paper~\cite{canetti2001universally}. An informal statement for guaranteeing that
a UC-secure protocol composes securely with arbitrary other protocols is the following:

\begin{theorem}[Universal Composition]
Let $\rho$, $\pi$, $\phi$ be protocols such that $\pi$ UC-emulates $\phi$. Then
protocol $\rho^{\phi \rightarrow \pi}$ UC-emulates protocol $\rho$.
\end{theorem}

\section{Proposed Work}

The goal of this work is to place the UC framework on a proper analytic
foundation by developing computer-aided tools and techniques for reasoning about
UC security. To achieve this goal, we first design and implement a programming
language called the Interactive Lambda Calculus (ILC for short) for constructing
algorithmic entities within the UC execution model. ILC is a functional
language, in the ML family of languages, with additional pi-calculus-like
constructs for modeling communication. Since processes in the UC framework are
modeled as interactive Turing machines, which are Turing machines that can write
on each other's tapes, we would like for ILC to be able to naturally express
these dialogues.

Perhaps most importantly, ILC's type system will enforce that any well-typed ILC
program will result in a valid execution with respect to the UC model of
execution. For example, in the UC execution model, only one process can be
active at a time---when a process $P_1$ writes to process $P_2$, $P_1$ passes its
activation to $P_2$. The precise semantics of the UC execution model, not the
mention the many variants of the UC framework (guc, juc, Symbolic UC, RSIM,
GNUC, etc.), can be difficult to keep track of. Thus, we leave it to a static
type checker to ensure that any program written in ILC will not violate the UC
execution model.

Since ILC is built with the UC execution model in mind, the next step is to
actually build the UC execution model in ILC. Pseudocode for the UC execution
model is in Figure~\ref{fig:execuc}. In addition, we will build a ``library'' of
algorithmic entities commonly seen in UC. We will start with common ideal
functionalities such as $\mathcal{F}_{\textsc{SMT}}$ (for secure message
transmission), $\mathcal{F}_{\textsc{AUTH}}$ (for authentication), and move onto
more complex functionalities for zero knowledge protocols and secure multiparty
computation protocols.

Finally, we will build a compiler for translating ILC programs into
EasyCrypt~\cite{barthe2011computer} modules. EasyCrypt is an automated tool for
elaborating cryptography proofs from sketches. Although the common reasoning
patterns for carrying out simulation-based proofs is not yet well supported in
EasyCrypt, we believe this will be a significant first step in mechanizing
simulation-based and UC proofs.

\section{SaUCy Execution}

\input{defn_execuc}

\section{ILC Language Definition}

\section{Functionalities in ILC}

\begin{func}[CRS]
    $\Func_{\textsc{CRS}}$ proceeds as follows, when parameterized by a distribution $D$.
    \begin{enumerate}
        \item When activated for the first time on input $(\texttt{value}, sid)$, choose a value $d \xleftarrow[]{R} D$ and send $d$ back to the activating party. In each other activation return the value $d$ to the activating party.
    \end{enumerate}
\end{func}

\begin{ilc}[CRS]
\lstinputlisting[style=ilc]{listings/F_crs.ilc}
\end{ilc}

\begin{func}[COM]
    $\Func_{\textsc{COM}}$ proceeds as follows, running with parties $P_1, \ldots, P_n$ and an adversary $S$.
    \begin{enumerate}
        \item Upon receiving a value $(\texttt{Commit}, sid, P_i, P_j, b)$ from $P_i$, where $b \in \{ 0, 1 \}$, record the value $b$ and send the message $(\texttt{Receipt}, sid, P_i, P_j)$ to $P_j$ and $S$. Ignore any subsequent \texttt{Commit} messages.

        \item Upon receiving a value $(\texttt{Open}, sid, P_i, P_j)$ from $P_i$, proceed as follows: If some value $b$ was previously recorded, then send the message $(\texttt{Open}, sid, P_i, P_j, b)$ to $P_j$ and $S$ and halt. Otherwise halt.
    \end{enumerate}
\end{func}

\begin{ilc}[COM]
\lstinputlisting[style=ilc]{listings/F_com.ilc}
\end{ilc}
    

\section{Related Work}

\section{Conclusion}

\bibliographystyle{plain}
\bibliography{saucy-report}

\appendix

\begin{figure}[htbp]
  \centering

\begin{grammar}
  Value Types
  & $A,B$
      &$\bnfas$&
      $x$ & Value variable
      \\ &&& $\bnfaltbrk \Unit$ & Unit value
      \\ &&& $\bnfaltbrk \Nat$         & Natural number
      \\ &&& $\bnfaltbrk A ** B$ & Product
      \\ &&& $\bnfaltbrk A + B$ & Sum type
      \\ &&& $\bnfaltbrk *! A$ & Intuitionistic type
      \\ &&& $\bnfaltbrk \tyRd{A}$ & Read channel
      \\ &&& $\bnfaltbrk \tyWr{A}$ & Write channel
      \\ &&& $\bnfaltbrk \tyU{C}$ & Thunk type
  \\[1ex]
  Computation Types
  & $C, D$
      &$\bnfas$ & 
             $A -> C$ & Value-consuming computation
      \\ &&& $\bnfaltbrk \tyF{A}$ & Value-producing computation
  \\[1ex]
  Linear Typing Contexts
  & $\Delta$
     &$\bnfas$& $\emptyctxt \bnfalt \Delta,x:A$
  \\
  Intuitionisitic Typing Contexts
  & $\Gamma$
     &$\bnfas$& $\emptyctxt \bnfalt \Gamma,x:A$
\end{grammar}

  \caption{Syntax of types and typing contexts}
  \label{fig:expr}
\end{figure}


\begin{figure}[htbp]
  \centering

\begin{grammar}
  Values
  & $v$
      &$\bnfas$&
      $x$
      \\ &&& $\bnfaltbrk \vUnit$ & Unit value
      \\ &&& $\bnfaltbrk n$         & Natural number
      \\ &&& $\bnfaltbrk \vPair{v_1}{v_2}$ & Pair of values
      \\ &&& $\bnfaltbrk \vInj{i}{v}$ & Injected value
      \\ &&& $\bnfaltbrk \vChan{c}$ & Channel (either read or write end)
      \\ &&& $\bnfaltbrk \vThunk{e}$ & Thunk (suspended, closed expression)
  \\[1ex]
  Expressions
  & $e$
      &$\bnfas$&
             $\Split{v}{x_1}{x_2}{e}$ & Pair elimination
      \\ &&& $\bnfaltbrk \Case{v}{x_1}{e_1}{x_2}{e_2}$ & Injection elimination
      \\ &&& $\bnfaltbrk \Ret{v}$ & Value-producing computation
      \\ &&& $\bnfaltbrk \Let{e_1}{x}{e_2}$ & Let-binding/sequencing
      \\ &&& $\bnfaltbrk \eApp{e}{v}$ & Function application
      \\ &&& $\bnfaltbrk \lam{x} e$ & Function abstraction
      \\ &&& $\bnfaltbrk \eForce{v}$ & Unsuspend (force) a thunk
      \\ &&& $\bnfaltbrk \eWr{v_1}{v_2}$ & Write channel~$v_1$ with value~$v_2$
      \\ &&& $\bnfaltbrk \eRd{v}$ & Read channel~$v$
      \\ &&& $\bnfaltbrk \eNu{x}{e}$ & Allocate channel as~$x$ in~$e$      \\ &&& $\bnfaltbrk e_1 *&& e_2$ & Fork~$e_1$, continue as~$e_2$
      \\ &&& $\bnfaltbrk e_1 *|| e_2$ & External choice between~$e_1$ and~$e_2$
\end{grammar}

  \caption{Syntax of values and expressions}
  \label{fig:expr}
\end{figure}


\begin{figure}[htbp]
{
  \centering

\begin{grammar}
  Modes & $m,n,p$ &$\bnfas$& $\Wm \bnfalt \Rm \bnfalt \Vm$ & (Write, Read and Value) 
\end{grammar}

\judgbox{m || n => p}{~~The parallel composition of modes $m$ and $n$ is mode~$p$.}
\begin{mathpar}
\Infer{sym}{m || n => p}{n || m => p}
\and \Infer{wv}{ }{\Wm || \Vm => \Wm}
\and \Infer{wr}{ }{\Wm || \Rm => \Wm}
\and \Infer{rr}{ }{\Rm || \Rm => \Rm}
\end{mathpar}
\\[2mm]
\judgbox{m ;; n => p}{~~The sequential composition of modes $m$ and $n$ is mode~$p$.}
\begin{mathpar}
\and \Infer{v$\ast$}{ }{\Vm ;; n => n}
\and \Infer{wv}{ }{\Wm ;; \Vm => \Wm}
\and \Infer{r$\ast$}{ }{\Rm ;; n => \Rm}
\and \Infer{wr}{ }{\Wm ;; \Rm => \Wm}
\end{mathpar}
}
Note that in particular, the following mode compositions are \emph{not derivable}:
\begin{itemize}
\item $\Wm || \Wm => p$ is \emph{not} derivable for any mode~$p$
\item $\Wm ;; \Wm => p$ is \emph{not} derivable for any mode~$p$
\end{itemize}
\caption{Syntax of modes; sequential and parallel mode composition.}
\label{fig:expr}
\end{figure}


\begin{figure}[htbp]
\centering
\judgbox{\Delta ; \Gamma |- e : C |> m}{~~Under $\Delta$ and $\Gamma$, expression~$e$ has type $C$ and mode $m$.}
\begin{mathpar}
%
\Infer{ret}
{\Delta ; \Gamma |- v : A}
{\Delta ; \Gamma |- \Ret{v} : \tyF A |> \Vm}
%
\and
%
\Infer{let}
{ m_1 ;; m_2 => m_3\\\\
\Delta_1        ; \Gamma |- e_1 : \tyF A |> m_1 \\\\
 \Delta_2, x:A ; \Gamma |- e_2 : C |> m_2
}
{\Delta_1, \Delta_2 ; \Gamma, x:A |- \Let{e_1}{x}{e_2} : C |> m_3}
%
\and
%
\Infer{ret!}
{\emptyctxt ; \Gamma |- v : A}
{\emptyctxt ; \Gamma |- \Ret{v} : \tyF (*! A) |> \Vm}
%
\and
%
\Infer{let!}
{\Delta_1 ; \Gamma |- v : *! A \\
 \Delta_2 ; \Gamma, x : A |- e : C |> m }
{\Delta_1, \Delta_2 ; \Gamma, x : A |- \LetBang{v}{x}{e} : C |> m}
%
\and
%
\Infer{lam}
{\Delta ; \Gamma |-         e :      C |> m}
{\Delta ; \Gamma |- \lam{x} e : A -> C |> m}
%
\and
%
\Infer{app}
{\Delta_1 ; \Gamma |- v : A \\
 \Delta_2 ; \Gamma |- e : A -> C |> m}
{\Delta_1, \Delta_2 ; \Gamma |- e\,v : C |> m}
%
\and
%
\Infer{nu}
{\Delta, x:\big(\tyRd A ** *!(\tyWr A)\big) ; \Gamma |- e : C |> m}
{\Delta                                 ; \Gamma |- \eNu{x}{e} : C |> m}
%
\\
%
\Infer{rd}
{\Delta; \Gamma |- v : \tyRd A}
{\Delta         |- \eRd{v} : \tyF (A ** (\tyRd A)) |> \Rm}
%
\and
%
\Infer{wr}
{\Delta_1; \Gamma   |- v_1 : \tyWr A \\
 \Delta_2; \Gamma   |- v_2 : A }
{\Delta_1, \Delta_2 |- \eWr{v_1}{v_2} : \tyF \Unit |> \Wm}
%
\\
%
\Infer{fork}
{
 m_1 || m_2 => m_3
 \\\\
 \Delta_1; \Gamma |- e_1 : C |> m_1
 \\\\
 \Delta_2; \Gamma |- e_2 : D |> m_2
}
{\Delta_1, \Delta_2 |- e_1 \xFork e_2 : D |> m_3}
%
\and
%
\Infer{choice}
{\Delta_1; \Gamma |- e_1 : C |> \Rm
           \\\\
 \Delta_2; \Gamma |- e_2 : C |> \Rm
}
{\Delta_1, \Delta_2 |- e_1 *|| e_2 : C |> \Rm}
%
\end{mathpar}
\end{figure}


\begin{figure}
\centering
\begin{grammar}
  Channels
  & $\Chans$ 
    & $\bnfas$ & $\emptyChans ~|~ \Chans, c$
    \\[2mm]
  Process pool
  & $\Procs$ 
    & $\bnfas$ & $\emptyProcs ~|~ \Procs, \proc$
    \\[2mm]
  Configurations
  & $C$
     & $\bnfas$ & $\Config{\Chans}{\Procs} $
     \\[2mm]
 Evaluation contexts
  & $E$
     & $\bnfas$ & $\Let{E}{x}{e}$
     \\ &&& $\bnfaltbrk \App{E}{v}$
     \\ &&& $\bnfaltbrk \bullet$
\\[2mm]
 Read contexts
  & $R$
     & $\bnfas$ & $\eRd{\vChan{c}} \oplus R$
     \\ &&& $\bnfaltbrk R \oplus \eRd{\vChan{c}}$
     \\ &&& $\bnfaltbrk \bullet$
\end{grammar}

\judgbox{e ---> e'}{~~Expression~$e_1$ reduces to~$e_2$.}
\begin{mathpar}
\Infer{let}
{}
{ \Let{\Ret{v}}{x}{e} ---> [v/x]e }
~~~
\Infer{app}
{}
{ \eApp{(\lam{x} e)}{v} ---> [v/x]e }
~~~
\Infer{force}
{ }
{ \eForce{\vThunk{e}} ---> e }
\and
\Infer{split}
{ }
{ \eSplit{\vPair{v_1}{v_2}}{x}{y}{e} ---> [v_1/x][v_2/y]e }
~~~
\Infer{case}
{ }
{ \eCase{\vInj{i}{v}}{x_1}{e_1}{x_2}{e_2} ---> e_i[v/x_i] }
\end{mathpar}

\judgbox{C_1 \equiv C_2}{~~Configurations~$C_1$ and $C_2$ are equivalent.}
\begin{mathpar}
\Infer{permProcs}
{  \Procs_1 \equiv_\textsf{perm} \Procs_2 }
{ \Config{\Chans}{\Procs_1} \equiv \Config{\Chans}{\Procs_2} }
\end{mathpar}

\judgbox{C_1 ---> C_2}{~~Configuration~$C_1$ reduces to $C_2$.}
\begin{mathpar}
\Infer{local}{ e ---> e' }
{ \Config{\Chans}{\Procs, E[e]} ---> \Config{\Chans}{\Procs, E[e]'} }
~~~
\Infer{fork}{ ~ }
{ \Config{\Chans}{\Procs, E[ e_1 \xFork e_2 ] } ---> \Config{\Chans}{\Procs, e_1, E[ e_2 ] } }
\and
\Infer{congr}{
C_1 \equiv C_1' 
\\
C_1' ---> C_2
\\
C_2 \equiv C_2'
}
{ C_1 ---> C_2' }
\and
\Infer{nu}{ c \notin \Chans }
{ \Config{\Chans}{\Procs,E[\eNu{x}{e}]} ---> \Config{\Chans, c}{\Procs, E[ [\vPair{\vChan{c}}{\vChan{c}} / x] e ]} }
\and
\Infer{rw}{ ~ }
{ \Config{\Chans}{\Procs,E_1[R[\eRd{\vChan{c}}] ],E_2[\eWr{\vChan{c}}{v}]} ---> \Config{\Chans}{\Procs,E_1[v],E_2[\vUnit]} }
\and
\end{mathpar}
\end{figure}

\end{document}
