\documentclass[a4paper]{article}
\usepackage[utf8]{inputenc}
\usepackage{lmodern}
\usepackage{amssymb}
\usepackage{amsthm}
\usepackage{mathtools}
\usepackage[margin=1in]{geometry}
\usepackage{listings}
\usepackage{algorithm}
\usepackage[noend]{algpseudocode}
\usepackage{framed}
\usepackage{listings}
\usepackage{xcolor}
\usepackage{textcomp}

\newcommand{\N}{\mathbb{N}}
\newcommand{\Q}{\mathbb{Q}}
\newcommand{\powerset}{\mathcal{P}}
\newcommand{\eq}{\equal}
\newcommand{\F}{\mathcal{F}}

\definecolor{mygreen}{rgb}{0.0, 0.5, 0.0}

\lstdefinestyle{ilc}
{
    numbers=left,
    stepnumber=1,
    captionpos=b,
    basicstyle = {\ttfamily},
    commentstyle = {\itshape\color{orange}},
    %morecomment=[s]{/*}{*/},
    stringstyle= {\color{purple}},
    keywordstyle = {\color{mygreen}},
    keywords={},
    keywordsprefix= {'},
    alsoletter= {' -},
    upquote = true,
    keywordstyle = [2]{\color{blue}},
    morekeywords = [2]{let, in, letrec, nu, wr, rd, ->},
    literate = {|}{{\textcolor{red}{|}}}1 {.|}{{\textcolor{red}{.|}}}1 {\&}{{\textcolor{red}{\texttt{\&}}}}1,
    frame=single,
    mathescape=true,
    showstringspaces=false
    numbersep=5pt,                   % how far the line-numbers are from the code
    numberstyle=\tiny, % the style that is used for the line-numbers
}

\begin{document}
\title{Functionalities in ILC}
\author{Cool kids}
\date{}
\maketitle

\section{Authenticated Message Transmission}

\begin{framed}
    \centerline{\textbf{Functionality} $\F_{\textsc{AUTH}}$}
    \begin{enumerate}
        \item Upon receiving an input $(\texttt{Send}, S, R, sid, m)$ from $ITI\ S$, generate a public delayed output $(\texttt{Sent}, S, sid, m)$ to $R$ and halt.
        \item Upon receiving $(\texttt{Corrupt-sender}, sid, m')$ from the adversary, and if the $(\texttt{Sent}, S, sid, m)$ output is not yet delivered to $R$, then output $(\texttt{Sent}, S, sid, m')$ to $R$ and halt.
    \end{enumerate}
\end{framed}

\lstinputlisting[style=ilc]{../F_auth.ilc}

\section{Secure Message Transmission}

\begin{framed}
    \centerline{\textbf{Functionality} $\F^l_{\textsc{SMT}}$}
    \begin{enumerate}
        \item Upon receiving an input $(\texttt{Send}, S, R, sid, m)$ from $ITI\ S$, send $(\texttt{Sent}, S, R, sid, l(m))$ to the adversary, generate a private delayed output $(\texttt{Sent}, S, sid, m)$ to $R$ and halt.
        \item Upon receiving $(\texttt{Corrupt}, sid, P)$ from the adversary, where $P \in \{S,R\}$, disclose $m$ to the adversary. Next, if the adversary provides a value $m'$, and $P = S$, and no output has been yet written to $R$, then output $(\texttt{Sent}, S, sid, m')$ to $R$ and halt.
    \end{enumerate}
\end{framed}

\lstinputlisting[style=ilc]{../F_smt.ilc}

\section{Zero Knowledge}

\begin{framed}
    \centerline{\textbf{Functionality} $\F_{\textsc{ZK(R)}}$}
    \begin{enumerate}
        \item Upon receiving input $(\texttt{Prove}, x, w, B)$ from $A$, leak $(A, B, x, R(x,w))$ to $S$.
        \item When $S$ returns $ok$, output $(\texttt{Verified}, A, x, R(x,w))$ to $B$.
    \end{enumerate}
\end{framed}


\lstinputlisting[style=ilc]{../F_zk.ilc}
\section{Key Exchange}

\begin{framed}
    \centerline{\textbf{Functionality} $\F_{\textsc{KE}}$}
    \begin{enumerate}
        \item Upon receiving input $(KE, B)$ from $A$, choose a key $k$ and leak $(A,B)$ to $S$.
        \item Upon receiving input $(KE, B)$ from $A$, leak $(A,B)$ to $S$.
        \item When $S$ returns $(ok, P)$ for $P=\{A,B\}$, output $(A,B,k)$ to $P$.
    \end{enumerate}
\end{framed}


\lstinputlisting[style=ilc]{../F_ke.ilc}

\section{Common Reference String}

From UC commitments~\cite{canetti2001universally}.

\begin{framed}
    \centerline{\textbf{Functionality} $\F_{\textsc{CRS}}$}
    \ \\
    \noindent $\F_{\textsc{CRS}}$ proceeds as follows, when parameterized by a distribution $D$.

    \begin{enumerate}
        \item When activated for the first time on input $(\texttt{value}, sid)$, choose a value $d \xleftarrow[]{R} D$ and send $d$ back to the activating party. In each other activation return the value $d$ to the activating party.
    \end{enumerate}
\end{framed}


\lstinputlisting[style=ilc]{../F_crs.ilc}

\section{Synchronous Communication}

\begin{framed}
    \centerline{\textbf{Functionality} $\F_{\textsc{SYN}}$}
    \ \\
    \noindent $\F_{\textsc{SYN}}$ expects its SID to be of the form $sid = (\mathcal{P}, sid')$ where $\mathcal{P}$ is a list of party identities among which sychronization is to be provided. It proceeds as follows.
    
    \begin{enumerate}
        \item At the first activation, initialize a round counter $r \leftarrow 1$, and send a public delayed output $(\texttt{Init}, sid)$ to all parties $\mathcal{P}$.

        \item Upon receiving input $(\texttt{Send}, sid, M)$ from a party $P \in \mathcal{P}$, where $M = \{(m_i, R_i)\}$ is a set of pairs of messages $m_i$ and recipient identities $R_i \in \mathcal{P}$, record $(P, M, r)$ and output $(sid, P, M, r)$ to the adversary.

        \item Upon receiving input $(\texttt{Receive}, sid, r')$ from a party $P \in \mathcal{P}$ do:
            \begin{enumerate}
                \item[(a)] If $r' = r$ (i.e.,\ $r'$ is the current round), and all uncorrupted parties in $\mathcal{P}$ have already sent their messages for round $r$, then:
                    \begin{enumerate}
                        \item[i.] Interpret $N$ as the list of messages sent by all parties in this round. That is, $N = \{(S_i, R_i, m_i)\}$ where each $S_i, R_i \in \mathcal{P}$, $m$ is a messages, and $S_i, R_i \in \mathcal{P}$. ($S_i$ is taken as the sender of message $m_i$ and $R_i$ is the receiver. Note that either sender or receiver may be corrupted.)
                        \item[ii.] Prepare for each party $P \in \mathcal{P}$ the list $L^r_P$ of messages that were sent to it in round $r$.
                        \item[iii.] Increment the rounder number: $r \leftarrow r + 1$.
                        \item[iv.] Output $(\texttt{Received}, sid, L^{r-1}_P)$ to $P$. (Let $L^0_P = \bot$.)
                    \end{enumerate}
                \item[(b)] If $r' < r$ then output $(\texttt{Received}, sid, L^{r'}_P)$ to $P$.
                \item[(c)] Else (i.e.,\ $r' > r$ or not all parties in $\mathcal{P}$ have sent their messages for round $r$), output $(\texttt{Round Incomplete})$ to $P$.
            \end{enumerate}
    \end{enumerate}
\end{framed}

\section{Commitments}

\begin{framed}
    \centerline{\textbf{Functionality} $\F_{\textsc{COM}}$}
    \ \\
    \noindent $\F_{\textsc{COM}}$ proceeds as follows, running with parties $P_1, \ldots, P_n$ and an adversary $S$.
    
    \begin{enumerate}
        \item Upon receiving a value $(\texttt{Commit}, sid, P_i, P_j, b)$ from $P_i$, where $b \in \{ 0, 1 \}$, record the value $b$ and send the message $(\texttt{Receipt}, sid, P_i, P_j)$ to $P_j$ and $S$. Ignore any subsequent \texttt{Commit} messages.

        \item Upon receiving a value $(\texttt{Open}, sid, P_i, P_j)$ from $P_i$, proceed as follows: If some value $b$ was previously recorded, then send the message $(\texttt{Open}, sid, P_i, P_j, b)$ to $P_j$ and $S$ and halt. Otherwise halt.
    \end{enumerate}
\end{framed}

\lstinputlisting[style=ilc]{../F_com.ilc}

\begin{framed}
    \centerline{\textbf{Functionality} $\F_{\textsc{MCOM}}$}
    \ \\
    \noindent $\F_{\textsc{MCOM}}$ proceeds as follows, running with parties $P_1, \ldots, P_n$ and an adversary $S$.
    
    \begin{enumerate}
        \item Upon receiving a value $(\texttt{Commit}, sid, cid, P_i, P_j, b)$ from $P_i$, where $b \in \{ 0, 1 \}$, record the tuple $(cid, P_i, P_j, b)$ and send the message $(\texttt{Receipt}, sid, cid, P_i, P_j)$ to $P_j$ and $S$. Ignore any subsequent $(\texttt{Commit}, sid, cid P_i, P_j, \ldots)$ messages.

        \item Upon receiving a value $(\texttt{Open}, sid, cid, P_i, P_j)$ from $P_i$, proceed as follows: If the tuple $(cid, P_i, P_j, b)$ is recorded, then send the message $(\texttt{Open}, sid, cid, P_i, P_j, b)$ to $P_j$ and $S$ and halt. Otherwise, do nothing.
    \end{enumerate}
\end{framed}

\lstinputlisting[style=ilc]{../F_mcom.ilc}

\section{Mix-Net}
From~\cite{wikstrom2004universally}.

\begin{framed}
    \centerline{\textbf{Functionality} $\F_{\textsc{MN}}$}
    \ \\
    \noindent $\F_{\textsc{MN}}$ running with mix-servers $M_1, \ldots, M_k$, senders $P_1, \ldots, P_N$, and ideal adversary $S$ proceeds as follows.
    
    \begin{enumerate}
        \item Initialize a list $L = \emptyset$, and set $J_P = \emptyset$ and $J_M = \emptyset$.
        \item Suppose $(P_i, \texttt{Send}, m_i)$, $m_i \in G_q$ is received from $C_\mathcal{I}$. If $i \not \in J_P$, set $J_P \leftarrow J_P \cup \{i\}$, and append $m_i$ to the list $L$. Then hand $(S, P_i, \texttt{Send})$ to $C_\mathcal{I}$.
        \item Suppose $(M_j, \texttt{Run})$ is received from $C_\mathcal{I}$. Set $J_M \leftarrow J_M \cup \{j\}$. If $|J_M| \geq k/2$, then sort the list $L$ lexicographically to form a list $L'$, and hand $((S, M_j, \texttt{Output}, L'), {\{(M_l, \texttt{Output}, L')\}}^{k}_{l=1})$ to $C_\mathcal{I}$. Otherwise, hand $C_\mathcal{I}$ the list $(S, M_j, \texttt{Run})$.
    \end{enumerate}
\end{framed}

\lstinputlisting[style=ilc]{../F_mn.ilc}

\section{Bulletin Board~\cite{wikstrom2004universally}}

\begin{framed}
    \centerline{\textbf{Functionality} $\F_{\textsc{BB}}$}
    \ \\
    \noindent $\F_{\textsc{BB}}$ running with participants $P_1, \ldots, P_k$ and ideal adversary $S$.
    
    \begin{enumerate}
        \item $\F_{\textsc{BB}}$ holds a database indexed on the integers. Initialize a counter $c = 0$.
        \item Upon receiving $(P_i, \texttt{Write}, m_i)$, $m_i \in {\{0, 1\}}^*$, from $C_\mathcal{I}$, store $(P_i, m_i)$ under the index $c$ in the database, hand $(S, \texttt{Write}, c, P_i, m_i)$ to $C_\mathcal{I}$, and set $c \leftarrow c + 1$.
        \item Upon receiving $(P_j, \texttt{Read}, c)$ from $C_\mathcal{I}$ check if a tuple $(P_i, m_i)$ is stored in the database under $c$. If so hand $((S, P_j, \texttt{Read}, c, P_i, m),(P_j, \texttt{Read}, c, P_i, m_i))$ to $C_\mathcal{I}$. If not, hand $((S, P_j, \texttt{NoRead},c), (P_j, \texttt{NoRead}, c))$ to $C_\mathcal{I}$.
    \end{enumerate}
\end{framed}

\lstinputlisting[style=ilc]{../F_bb.ilc}

\section{Distributed El Gamal Key Generation~\cite{wikstrom2004universally}}

\begin{framed}
    \centerline{\textbf{Functionality} $\F_{\textsc{KG}}$}
    \ \\
    \noindent $\F_{\textsc{KG}}$ running with mix-servers $M_1, \ldots, M_k$, senders $P_1, \ldots, P_N$, and ideal adversary $S$ proceeds as follows.
    
    \begin{enumerate}
        \item Initialize sets $J_j = \emptyset$ for $j = 0, \ldots, k$.
        \item Until $|J_0| = k$, wait for $(M_j, \texttt{MyKey}, x_j, y_j)$ from $C_\mathcal{I}$ such that $x_j \in \mathbb{Z}_q$, $y_j = g^{x_j}$, and $j \not \in J_0$. Set $J_0 \leftarrow J_0 \cup \{j\}$.
        \item Hand $((S, \texttt{PublicKeys}, y_1, \ldots, y_k), {\{(P_j, \texttt{PublicKeys}, y_1, \ldots, y_k)\}}^{N}_{j=1},$ ${\{(M_j, \texttt{Keys}, x_j, y_1, \ldots, y_k)\}}^{k}_{j=1})$ to $C_\mathcal{I}$.
        \item If $(M_j, \texttt{Recover}, M_l)$ is received from $C_\mathcal{I}$, set $J_l \leftarrow J_l \cup \{j\}$. If $|J_l| \geq k/2$, then hand $((S, \texttt{Recovered}, M_l, x_l), {\{(M_j, \texttt{Recovered}, M_l, x_l)\}}^{k}_{j=1})$ to $C_\mathcal{I}$, and otherwise hand $(S, M_j, \texttt{Recover}, M_l)$ to $C_\mathcal{I}$.
    \end{enumerate}
\end{framed}


\section{Zero-Knowledge Proof of Knowledge~\cite{wikstrom2004universally}}

\begin{framed}
    \centerline{\textbf{Functionality} $\F_{\textsc{ZK(R)}}$}
    \ \\
    \noindent $\F_{\textsc{KG}}$ of a witness $w$ to an element $x \in \mathcal{L}$, running with provers $P_1, \ldots, P_N$, and verifiers $M_1, \ldots, M_k$ proceeds as follows.
    
    \begin{enumerate}
        \item Upon receipt of $(P_i, \texttt{Prover}, x, w)$ from $C_\mathcal{I}$, store $w$ under the tag $(P_i, x)$, and hand $(S, P_i, \texttt{Prover}, x, R(x, w))$ to $C_\mathcal{I}$. Ignore further messages from $P_i$.
        \item Upon receipt of $(M_j, \texttt{Question}, P_i, x)$ from $C_\mathcal{I}$, let $w$ be the string stored under the tag $(P_i, x)$ (the empty string if nothing is stored), and hand $((S, M_j, \texttt{Verifier}, P_i, x, R(x, w)), (M_j, \texttt{Verifier}, P_i, R(x, w)))$ to $C_\mathcal{I}$.
    \end{enumerate}
\end{framed}


\section{SaUCY execution model}
\lstinputlisting[style=ilc]{../SaUCy/send_delay.ilc}

\bibliography{ilc-funcs}
\bibliographystyle{plain}
\end{document}
