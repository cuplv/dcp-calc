\documentclass[a4paper]{article}
\usepackage[utf8]{inputenc}
\usepackage{lmodern}
\usepackage{amssymb}
\usepackage{amsthm}
\usepackage{mathtools}
\usepackage[margin=1in]{geometry}
\usepackage{listings}
\usepackage{algorithm}
\usepackage[noend]{algpseudocode}
\usepackage{framed}
\usepackage{listings}
\usepackage{xcolor}
\usepackage{textcomp}

\newcommand{\N}{\mathbb{N}}
\newcommand{\Q}{\mathbb{Q}}
\newcommand{\powerset}{\mathcal{P}}
\newcommand{\eq}{\equal}
\newcommand{\F}{\mathcal{F}}

\definecolor{mygreen}{rgb}{0.0, 0.5, 0.0}

\lstdefinestyle{ilc}
{
    numbers=left,
    stepnumber=1,
    captionpos=b,
    basicstyle = {\ttfamily},
    commentstyle = {\itshape\color{orange}},
    stringstyle= {\color{purple}},
    keywordstyle = {\color{mygreen}},
    keywords={},
    keywordsprefix= {'},
    alsoletter= {' -},
    upquote = true,
    keywordstyle = [2]{\color{blue}},
    morekeywords = [2]{let, in, letrec, nu, wr, rd, ->},
    literate = {|}{{\textcolor{red}{|}}}1 {.|}{{\textcolor{red}{.|}}}1 {\&}{{\textcolor{red}{\texttt{\&}}}}1,
    frame=single,
    mathescape=true,
    showstringspaces=false
    numbersep=5pt,                   % how far the line-numbers are from the code
    numberstyle=\tiny, % the style that is used for the line-numbers
}

\begin{document}
\title{Functionalities in ILC}
\author{Cool kids}
\date{}
\maketitle

\section{Authenticated Message Transmission}

\begin{framed}
    \centerline{\textbf{Functionality} $\F_{\textsc{AUTH}}$}
    \begin{enumerate}
        \item Upon receiving an input $(\texttt{Send}, S, R, sid, m)$ from $ITI\ S$, generate a public delayed output $(\texttt{Sent}, S, sid, m)$ to $R$ and halt.
        \item Upon receiving $(\texttt{Corrupt-sender}, sid, m')$ from the adversary, and if the $(\texttt{Sent}, S, sid, m)$ output is not yet delivered to $R$, then output $(\texttt{Sent}, S, sid, m')$ to $R$ and halt.
    \end{enumerate}
\end{framed}


\lstinputlisting[style=ilc]{../F_auth.ilc}

\section{Secure Message Transmission}

\begin{framed}
    \centerline{\textbf{Functionality} $\F^l_{\textsc{SMT}}$}
    \begin{enumerate}
        \item Upon receiving an input $(\texttt{Send}, S, R, sid, m)$ from $ITI\ S$, send $(\texttt{Sent}, S, R, sid, l(m))$ to the adversary, generate a private delayed output $(\texttt{Sent}, S, sid, m)$ to $R$ and halt.
        \item Upon receiving $(\texttt{Corrupt}, sid, P)$ from the adversary, where $P \in \{S,R\}$, disclose $m$ to the adversary. Next, if the adversary provides a value $m'$, and $P = S$, and no output has been yet written to $R$, then output $(\texttt{Sent}, S, sid, m')$ to $R$ and halt.
    \end{enumerate}
\end{framed}

\lstinputlisting[style=ilc]{../F_smt.ilc}

\section{Zero Knowledge Proofs}

\begin{framed}
    \centerline{\textbf{Functionality} $\F_{\textsc{ZK(R)}}$}
    \begin{enumerate}
        \item Upon receiving input $(\texttt{Prove}, x, w, B)$ from $A$, leak $(A, B, x, R(x,w))$ to $S$.
        \item When $S$ returns $ok$, output $(\texttt{Verified}, A, x, R(x,w))$ to $B$.
    \end{enumerate}
\end{framed}


\lstinputlisting[style=ilc]{../F_zk.ilc}
\section{Key Exchange}

\begin{framed}
    \centerline{\textbf{Functionality} $\F_{\textsc{KE}}$}
    \begin{enumerate}
        \item Upon receiving input $(KE, B)$ from $A$, choose a key $k$ and leak $(A,B)$ to $S$.
        \item Upon receiving input $(KE, B)$ from $A$, leak $(A,B)$ to $S$.
        \item When $S$ returns $(ok, P)$ for $P=\{A,B\}$, output $(A,B,k)$ to $P$.
    \end{enumerate}
\end{framed}


\lstinputlisting[style=ilc]{../F_ke.ilc}

\section{Common Reference String}

From UC commitments~\cite{canetti2001universally}.

\begin{framed}
    \centerline{\textbf{Functionality} $\F_{\textsc{CRS}}$}
    \ \\
    \noindent $\F_{\textsc{CRS}}$ proceeds as follows, when parameterized by a distribution $D$.

    \begin{enumerate}
        \item When activated for the first time on input $(\texttt{value}, sid)$, choose a value $d \xleftarrow[]{R} D$ and send $d$ back to the activating party. In each other activation return the value $d$ to the activating party.
    \end{enumerate}
\end{framed}


\lstinputlisting[style=ilc]{../F_crs.ilc}

\section{Synchronous Communication}

\begin{framed}
    \centerline{\textbf{Functionality} $\F_{\textsc{SYN}}$}
    \ \\
    \noindent $\F_{\textsc{SYN}}$ expects its SID to be of the form $sid = (\mathcal{P}, sid')$ where $\mathcal{P}$ is a list of party identities among which sychronization is to be provided. It proceeds as follows.
    
    \begin{enumerate}
        \item At the first activation, initialize a round counter $r \leftarrow 1$, and send a public delayed output $(\texttt{Init}, sid)$ to all parties $\mathcal{P}$.

        \item Upon receiving input $(\texttt{Send}, sid, M)$ from a party $P \in \mathcal{P}$, where $M = \{(m_i, R_i)\}$ is a set of pairs of messages $m_i$ and recipient identities $R_i \in \mathcal{P}$, record $(P, M, r)$ and output $(sid, P, M, r)$ to the adversary.

        \item Upon receiving input $(\texttt{Receive}, sid, r')$ from a party $P \in \mathcal{P}$ do:
            \begin{enumerate}
                \item[(a)] If $r' = r$ (i.e.,\ $r'$ is the current round), and all uncorrupted parties in $\mathcal{P}$ have already sent their messages for round $r$, then:
                    \begin{enumerate}
                        \item[i.] Interpret $N$ as the list of messages sent by all parties in this round. That is, $N = \{(S_i, R_i, m_i)\}$ where each $S_i, R_i \in \mathcal{P}$, $m$ is a messages, and $S_i, R_i \in \mathcal{P}$. ($S_i$ is taken as the sender of message $m_i$ and $R_i$ is the receiver. Note that either sender or receiver may be corrupted.)
                        \item[ii.] Prepare for each party $P \in \mathcal{P}$ the list $L^r_P$ of messages that were sent to it in round $r$.
                        \item[iii.] Increment the rounder number: $r \leftarrow r + 1$.
                        \item[iv.] Output $(\texttt{Received}, sid, L^{r-1}_P)$ to $P$. (Let $L^0_P = \bot$.)
                    \end{enumerate}
                \item[(b)] If $r' < r$ then output $(\texttt{Received}, sid, L^{r'}_P)$ to $P$.
                \item[(c)] Else (i.e.,\ $r' > r$ or not all parties in $\mathcal{P}$ have sent their messages for round $r$), output $(\texttt{Round Incomplete})$ to $P$.
            \end{enumerate}
    \end{enumerate}
\end{framed}

\section{Commitments}

\begin{framed}
    \centerline{\textbf{Functionality} $\F_{\textsc{COM}}$}
    \ \\
    \noindent $\F_{\textsc{COM}}$ proceeds as follows, running with parties $P_1, \ldots, P_n$ and an adversary $S$.
    
    \begin{enumerate}
        \item Upon receiving a value $(\texttt{Commit}, sid, P_i, P_j, b)$ from $P_i$, where $b \in \{ 0, 1 \}$, record the value $b$ and send the message $(\texttt{Receipt}, sid, P_i, P_j)$ to $P_j$ and $S$. Ignore any subsequent \texttt{Commit} messages.

        \item Upon receiving a value $(\texttt{Open}, sid, P_i, P_j)$ from $P_i$, proceed as follows: If some value $b$ was previously recorded, then send the message $(\texttt{Open}, sid, P_i, P_j, b)$ to $P_j$ and $S$ and halt. Otherwise halt.
    \end{enumerate}
\end{framed}

\begin{framed}
    \centerline{\textbf{Functionality} $\F_{\textsc{MCOM}}$}
    \ \\
    \noindent $\F_{\textsc{MCOM}}$ proceeds as follows, running with parties $P_1, \ldots, P_n$ and an adversary $S$.
    
    \begin{enumerate}
        \item Upon receiving a value $(\texttt{Commit}, sid, cid, P_i, P_j, b)$ from $P_i$, where $b \in \{ 0, 1 \}$, record the tuple $(cid, P_i, P_j, b)$ and send the message $(\texttt{Receipt}, sid, cid, P_i, P_j)$ to $P_j$ and $S$. Ignore any subsequent $(\texttt{Commit}, sid, cid P_i, P_j, \ldots)$ messages.

        \item Upon receiving a value $(\texttt{Open}, sid, cid, P_i, P_j)$ from $P_i$, proceed as follows: If the tuple $(cid, P_i, P_j, b)$ is recorded, then send the message $(\texttt{Open}, sid, cid, P_i, P_j, b)$ to $P_j$ and $S$ and halt. Otherwise, do nothing.
    \end{enumerate}
\end{framed}


\bibliography{ilc-funcs}
\bibliographystyle{plain}
\end{document}
