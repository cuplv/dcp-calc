\usepackage{semantic}   % Tools for typesetting PL semantics
\usepackage{braket}     % Easy angle-bracket notation
\usepackage{mathpartir} % Used to typeset blocks of inference rules
\usepackage{rotating} % for sidewaysfigure
%\usepackage{proof}
\usepackage{pdflscape}
\usepackage{fancyvrb} % to use \verb in footnotes
\usepackage{stmaryrd}
%\usepackage{amsmath,amsthm,amssymb}
%\usepackage{thmtools,thm-restate}
\usepackage{mathtools}

% Marvosym
\let\MathRightArrow\Rightarrow % save original definition of \Rightarrow
\usepackage{marvosym} % feloniously overrides \Rightarrow
\def\Rightarrow{\MathRightArrow}

%% \declaretheorem[style=mytheoremstyle]{theorem}
%% \declaretheorem[style=mytheoremstyle]{property}
%% %\declaretheorem[style=mytheoremstyle, sibling=theorem]{lemma}
%% \declaretheorem[style=mytheoremstyle]{lemma}
%% \declaretheorem[style=mytheoremstyle, sibling=lemma]{corollary}
%% \declaretheorem[style=mytheoremstyle, sibling=lemma]{conjecture}
%% \declaretheorem{example}
%% \makeatletter
%% \declaretheorem[style=mytheoremstyle, sibling=lemma, 
%% %             postheadhook={%
%% %               envname is `\thmt@envname'; %
%% %               thmname is `\thmt@thmname'; %
%% %               optarg is `\thmt@optarg'; %
%% %%               innercounters are `\thmt@innercounters'.
%% %             }
%%        ]{proposition}
%% \makeatother
%% \declaretheorem{remark}
%% \declaretheorem[style=mytheoremstyle]{definition}


%% \theoremstyle{remark}

\newcommand{\calc}{$\lambda_{\textsf{nom}}$\xspace}

\newcommand{\Label}[1]{\label{#1}} % XXX
\newcommand{\FLabel}[1]{\label{#1}} % XXX
\newcommand{\D}{\mathcal{D}}
\newcommand{\Ss}{\mathcal{S}}
\newcommand{\derives}{::}
\newcommand{\satisfactory}{~\textsf{satisfactory}}
%\newcommand{\wf}{~\textsf{wf}}

%% \newtheorem{thm}{Theorem}[section]
%% \newtheorem*{thm*}{Theorem}
%% \newtheorem{lem}[thm]{Lemma}
%% \newtheorem{conj}[thm]{Conjecture}
%% \newtheorem{prop}[thm]{Proposition}
%% \newtheorem*{cor}{Corollary}
%% \newcommand{\thmref}[1]{Theorem~\ref{thm:#1}}
%% \newcommand{\thmreftwo}[2]{Theorems \ref{thm:#1} and~\ref{thm:#2}}
%% \newtheorem{defn}[thm]{Definition}

\newcommand{\type}{\star}
\newcommand{\sort}{\gamma}
\newcommand{\namesort}{\boldsf{Nm}}
\newcommand{\namesetsort}{\boldsf{NmSet}}
\newcommand{\unitsort}{\boldsf{1}}

\newcommand{\Nmsp}{\tyname{Nmsp}}

\newcommand{\xwfeff}{\text{wf-effects}}
\newcommand{\wfeff}{~\xwfeff}
\newcommand{\xwfname}{\text{wf-name}}
\newcommand{\wfname}{~\xwfname}

\newcommand{\xwf}{\text{wf}}
\newcommand{\wf}{~\xwf}

\newcommand{\xwfgraph}{\xwf}
\newcommand{\wfgraph}{~\xwfgraph}

% \newcommand{\ixf}{\boldsf{f}}

\newcommand{\namevar}{\nu}

\newcommand{\Unit}{\tyname{unit}}
\newcommand{\unitty}{\Unit}
\newcommand{\unit}{\textvtt{()}}
\newcommand{\unitexp}{\unit}

\newcommand{\unitindex}{\unit}

\newcommand{\runonboldsf}{\sffamily\bfseries\selectfont}
\newcommand{\boldsf}[1]{\text{\runonboldsf #1}}

\newcommand{\xF}{\boldsf{F}}
\newcommand{\F}{\xF\,}
\newcommand{\xU}{\boldsf{U}}
\newcommand{\U}{\xU\,}

\newcommand{\disj}{\mathop{\bot}}

\newcommand{\trueprop}{\boldsf{tt}}
\newcommand{\andpropsym}{\boldsf{and}}
\newcommand{\andprop}{\mathrel{\andpropsym}}
\newcommand{\impty}{\supset}

% \Univ currently deprecated
\newcommand{\Univsym}{\Pi}
% \Univ currently deprecated
\newcommand{\Univ}[1]{\Univsym{#1}.\,}

% Currently, using \All for both type variables (\All{\alpha : K}
% and index variables (\All{ea : \sort})
\newcommand{\Allsym}{\forall}
\newcommand{\All}[1]{\Allsym{#1}.\,}

% Name function arrow (name in scope at both the type and term level!)
% \Namearr{a}{... @> ...}
\newcommand{\Namearrsym}{\aleph}
\newcommand{\Namearr}[2]{\Namearrsym{#1}:{#2} \arr}

\newcommand{\namearrintro}[1]{\Namearrsym{#1}.\;}
\newcommand{\namearrelimsym}{\texttt{@}}
\newcommand{\namearrelim}[2]{{#1} \mathop{\namearrelimsym} {#2}}

\newcommand{\tyapp}[2]{{#1}\tbrack{#2}}
% \newcommand{\idxapp}[2]{{#1}\tbrack{#2}}
\newcommand{\idxapp}[2]{{#1}({#2})}

\newcommand{\Split}[4]{\keyword{split}\Lparen{#1}, {#2}.{#3}.{#4}\Rparen}
\newcommand{\Case}[5]{\keyword{case}\Lparen{#1}, {#2}.{#3}, {#4}.{#5}\Rparen}

\newcommand{\inj}[1]{\keyword{inj}_{#1}\,}
\newcommand{\Inj}[1]{\inj{#1}}

\newcommand{\Scope}[2]{\keyword{scope}\Lparen{#1}, {#2}\Rparen}

\newcommand{\Sig}{\mathcal{S}}


%% Math ligatures (thanks to the semantic package) that make it
%% easier to typeset math using readable LaTeX text.
%\mathlig{|-->}{\longmapsto}
\mathlig{::=}{\bnfas}
\mathlig{|}{\;|\;}
% \mathlig{[[}{\mbsf{[}}
% \mathlig{]]}{\mbsf{]}}
\mathlig{[[}{\texttt{\upshape[}}
\mathlig{]]}{\texttt{\upshape]}}
\mathlig{**}{\times}
\mathlig{|>}{\rhd}
\mathlig{->}{\arr}
\mathlig{-->}{\impty}
\mathlig{=>}{\Rightarrow}

\mathlig{@>}{\arr_{\namesort}}


\newcommand{\eff}[2]{{#1} |> {#2}}
\newcommand{\effseqsym}{\mathsf{then}}
\newcommand{\effseq}{\,\effseqsym\,}

\newcommand{\effcoalsym}{\mathsf{after}}
\newcommand{\effcoal}{\,\effcoalsym\,}

\newcommand{\Lparen}{\texttt{(}}
\newcommand{\Rparen}{\texttt{)}}
% \newcommand{\Pair}[2]{\Lparen{#1}\texttt{,}{#2}\Rparen}

%\mathlig{unit}{\unitty}


% Bidirectional typing

\newcommand{\chkcolor}{dBlue}
\newcommand{\syncolor}{dRed}
\newcommand{\isyncolor}{dGreen}
\newcommand{\chk}{\mathrel{\mathcolor{\chkcolor}{\Leftarrow}}}
\newcommand{\uncoloredsyn}{{\Rightarrow}}
\newcommand{\syn}{\mathrel{\mathcolor{\syncolor}{\uncoloredsyn}}}
\newcommand{\RRightarrow}{\mathrel{\mathrlap{\Rightarrow}\mkern2mu\Rightarrow}}
\newcommand{\isyn}{\mathrel{\mathcolor{\isyncolor}{\RRightarrow}}}
\mathlig{<<}{\langle}
\mathlig{>>}{\rangle}
\mathlig{==>}{\syn}
\mathlig{<==}{\chk}
\mathlig{||}{\Downarrow}
\mathlig{!!}{\Downarrow}
% \mathlig{..}{\epsilon}

\newcommand{\e}{\epsilon}
\newcommand{\hist}{\mathcal{H}}

\newcommand{\ambns}{M}

\newcommand{\disjoint}{\mathrel{\bot}}

\newcommand{\tbrack}[1]{\texttt{\upshape[}{#1}\texttt{\upshape]}}

\newcommand{\rootname}{\textvtt{root}}

%\newcommand{\tv}{tv}
\newcommand{\Mv}{V}
\newcommand{\ntevalsym}{\Downarrow_{\mathsf{M}}}
\newcommand{\nteval}{\mathrel{\ntevalsym}}

\newcommand{\xrefv}{\keyword{ref}}
\newcommand{\xthunk}{\keyword{thunk}}
\newcommand{\xunthunk}{\keyword{unthunk}}
\newcommand{\xname}{\keyword{name}}
%\newcommand{\xName}{\tyname{Name}}
\newcommand{\xName}{\tyname{Nm}}

\newcommand{\xRef}{\tyname{Ref}}
%\newcommand{\xThunk}{\tyname{Thunk}}
\newcommand{\xThunk}{\tyname{Thk}}
\newcommand{\xUnthunk}{\tyname{Unthunk}}

\newcommand{\refv}[1]{\xrefv\;{#1}}
\newcommand{\thunk}[1]{\xthunk\;{#1}}
\newcommand{\unthunk}[1]{\xunthunk\;{#1}}
\newcommand{\name}[1]{\xname\;{#1}}
\newcommand{\Name}[1]{\xName\tbrack{#1}}


\newcommand{\Ref}[1]{\xRef\tbrack{#1}\,}
\newcommand{\Thk}[1]{\xThunk\tbrack{#1}\,}
\newcommand{\Unthk}[1]{\xUnthunk\tbrack{#1}\,}
\newcommand{\Thunk}[2]{\keyword{thunk}\Lparen{#1},{#2}\Rparen}
\newcommand{\Unthunk}[1]{\keyword{unthunk}\Lparen{#1}\Rparen}
\newcommand{\Refe}[2]{\keyword{ref}\Lparen{#1},{#2}\Rparen} % Ref expression form

\newcommand{\List}[3]{\tyname{List}\tbrack{#1;#2}~{#3}}
\newcommand{\Art}[2]{\tyname{Art}\tbrack{#1}~{#2}}
\newcommand{\Int}[0]{\tyname{Int}}

\let\Force\undefined
\newcommand{\Force}[1]{\keyword{force}\Lparen{#1}\Rparen}
\newcommand{\Get}[1]{\keyword{get}\Lparen{#1}\Rparen}

\newcommand{\Ns}[2]{\keyword{ns}\Lparen{#1}\Rparen\{#2\}}

\newcommand{\Ret}[1]{\keyword{ret}\Lparen{#1}\Rparen}
\newcommand{\Let}[3]{\keyword{let}\Lparen{#1},{#2}.{#3}\Rparen}

\newcommand{\xstoretype}{\textsf{has-store-type}}
\newcommand{\storetype}{\;\xstoretype\;}
\newcommand{\xisname}{\textsf{is-name}}
\newcommand{\isname}{\;\xisname}

\newcommand{\PreSt}[3]{{#1}\vdash^{#2}_{#3}}

\newcommand{\xterminal}{\textsf{terminal}}
\newcommand{\terminal}{\;\xterminal}

\newcommand{\nscons}[2]{\left[\!\left[ #1~(#2) \right]\!\right]} % Namespace/Name cons operation
% \newcommand{\nsEval}[3]{\vdash t\;p !! q}
% use \nteval (infix) instead

\newcommand{\emptygraph}{\varepsilon}

\newcommand{\bnfheader}[1]{\multicolumn{4}{@{}l}{\text{#1}}}
\newenvironment{sbnfarray}{\begin{tabular}{c}\begin{array}[t]{lr@{~~}c@{~~~}ll}}{\end{array}\end{tabular}}
\newenvironment{bnfarray}{\begin{tabular}{c}\begin{array}[t]{r@{~~}c@{~~~}ll}}{\end{array}\end{tabular}}
\newenvironment{xbnfarray}{\begin{tabular}{c}\begin{array}[t]{r@{~~}c@{~}l@{~~}l}}{\end{array}\end{tabular}}

\newcommand{\DHasEffects}[3]{#1~\textsf{reads}~#2~\textsf{writes}~#3}
\newcommand{\DByHasEffects}[4]{#1~\textsf{by}~#2~\textsf{reads}~#3~\textsf{writes}~#4}
\newcommand{\mergeRds}{\cup}
\newcommand{\joinWrs}{\mathrel{\ast}}
\newcommand{\joinGrs}{\mathrel{\ast}}
\newcommand{\mergeGrs}{\cup}
\newcommand{\restrict}[2]{\textsf{restrict}\lfloor{#1}\rfloor_{#2}}
\newcommand{\dom}[1]{\textsf{dom}(#1)}
\newcommand{\update}[4]{#1\{#2 \mapsto #3\} = #4}
%\newcommand{\disjoint}{\cupplus}

%\newcommand{\Dee}{\mathcal{D}}
%\newcommand{\D}{\mathcal{\Dee}}

\newcommand{\te}{e_{\mathsf{terminal}}}

\newcommand{\nfap}[3]{{\color{blue}{}^{#1}_{#2}}{\color{blue}\big[} #3 {\color{blue}\big]}}

%%% Local Variables: 
%%% mode: latex
%%% TeX-master: types
%%% End: 

